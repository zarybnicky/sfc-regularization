% Created 2019-12-18 Wed 13:53
% Intended LaTeX compiler: pdflatex
\documentclass[presentation]{beamer}
\usepackage[utf8x]{inputenc}
\usepackage[T1]{fontenc}
\usepackage{graphicx}
\usepackage{grffile}
\usepackage{longtable}
\usepackage{wrapfig}
\usepackage{rotating}
\usepackage[normalem]{ulem}
\usepackage{amsmath}
\usepackage{textcomp}
\usepackage{amssymb}
\usepackage{capt-of}
\usepackage{hyperref}
\usepackage{minted}
\mode<beamer>{\usetheme{Madrid}}
\usetheme{default}
\author{Jakub Zárybnický}
\date{18. 12. 2019}
\title{Regularizace v neuronových sítích}
\hypersetup{
 pdfauthor={Jakub Zárybnický},
 pdftitle={Regularizace v neuronových sítích},
 pdfkeywords={},
 pdfsubject={},
 pdfcreator={Emacs 26.1 (Org mode 9.1.9)}, 
 pdflang={Czech}}
\begin{document}

\maketitle

\begin{frame}[label={sec:orgbbc10c9}]{Zadání}
\begin{itemize}
\item demonstrovat regularizaci při učení neuronových sítí
\item regularizace
\begin{itemize}
\item = omezení velikosti jednotlivých váhových koeficientů
\item jeden ze způsobů zamezování problému přeučení (overfitting) při učení NN s učitelem
\item L1 (lasso regression) - zmenšuje váhové vektory o konstantu v každém kroku
\item L2 (ridge regression) - zmenšuje váhové vektory proporčně k jejich velikosti
\end{itemize}
\end{itemize}
\end{frame}

\begin{frame}[label={sec:orgf510ca8}]{Přístup}
\begin{itemize}
\item záměr = vytvořit NN od základů pomocí násobení matic v Numpy
\begin{itemize}
\item (stejně jako v projektech ZZN a SUI)
\end{itemize}
\item v Javě:
\begin{itemize}
\item knihovna Nd4j (nd-array, deeplearning4java)
\item nejbližší k Numpy v Pythonu
\end{itemize}
\end{itemize}
\end{frame}

\begin{frame}[label={sec:org77b6584}]{Výsledek}
\begin{itemize}
\item implementace libovolně tvarovaných NN s L1/L2/L1+L2 regularizací
\item jednoduchý program pro učení NN s regularizací
\item konfigurace změnami v kódu
\item vizualizace v terminálu nebo ručním exportováním dat
\end{itemize}
\end{frame}

\begin{frame}[label={sec:orge94cc45}]{Chyby při implementaci:}
\begin{itemize}
\item chyba v definici \emph{cost function}
\item pomalé - iterace přes vzorky
\item důvěra v \emph{data science} ekosystém Javy
\begin{itemize}
\item Nd4j - jednoúčelová knihovna (2D a 3D pole, 1D téměř nepodporované)
\item žádný jednoduchý způsob vizualizace (\textasciitilde{} matplotlib)
\end{itemize}
\item téměř nulový přenos zkušeností z velmi podobných projektů v ZZN a SUI
\end{itemize}
\end{frame}
\end{document}